\documentclass[options]{article}
\begin{document}
\textbf{How to permanently delete a Facebook account}

{\textbf{Prepared by: Rukundo Jonathan, 215014765, 15/U/12435/EVE
}}

\section{\textbf{Introduction.}} 
Facebook makes it clear in its Terms of Service that you are surrendering ownership rights to the intellectual property (i.e., your updates and photos) that you upload to the platform. This has in recent years seen an increasing number of users drop their Facebook accounts.
However, the process of permanently deleting your account is anything but intuitive. This report lays out all the details that you may need in order to delete your Facebook account for good.

\section{\textbf{Log in to Facebook}} 
You can't completely delete your account without logging in to Facebook first. If you could, it would be much easier for people with malicious intent to delete unwitting users' Facebook accounts. To begin the process of account deletion, visit Facebook.com and provide your login info as you normally would.

\subsection{\textbf{Save any data you need}}
Before you delete your account, save anything from your account that there is chance you will need later. Depending on your needs, this may include contact information, photos, personal profile content, and more. You will lose access to all of this when you delete your account, so now is your chance to make a backup of any important data.

\subsection{\textbf{Visit the "Delete Account" page}}
This page, which allows you to permanently delete your account, isn't easy to find. It cannot be accessed through Facebook normally — you have to search for it on a search engine or look it up on Facebook's help page, Facebook.com/help. 
A small window should pop up asking you to enter your account password and provide the letters in the random CAPTCHA image to prove that you're human. Once you have entered these, press “Okay” to schedule your account for permanent deletion.

\subsection{\textbf{Wait 14 days}} 
Once you complete the account deletion process, your account is immediately deactivated. This means that your timeline immediately disappears from the site and other users are no longer able to search for you. However, your account isn't permanently deleted for 14 days. During this time, if you log in to your account, you'll have the option to restore it. After the 14 days have passed, your account is permanently deleted and cannot be recovered by any means, so if you're having second thoughts, act fast!
If you're positive that you want your account permanently deleted, no further action is required on your part. Simply wait 14 days without logging in and your account will be gone for good.

\section{\textbf{Conclusion}}
While permanent deletion will delete almost everything about your account, certain things will remain even after your account is deleted. Most importantly, deleting your account won't delete the personal messages that you've sent to other users. Though these users won't be able to search for you, if they manually scroll through the messages in their inbox, they can still see the messages you've sent. However, the message will be labelled with the greyed out name "Facebook User" (rather than your name) and the profile picture will be set to the default silhouette.
In addition, it's worth noting that, while your account may be gone, Facebook retains the right to keep certain types of information even after your account is deleted. Certain types of data, like photos, are deleted from Facebook's servers soon after the account is deleted.

\section{\textbf{Recommendation}}
If you opt to permanently delete your Facebook account, please keep in mind that you won't be able to reactivate your account or retrieve anything you've added. Thus it’s always wise to download a copy of your info from Facebook.


\end{document}